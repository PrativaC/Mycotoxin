% Options for packages loaded elsewhere
\PassOptionsToPackage{unicode}{hyperref}
\PassOptionsToPackage{hyphens}{url}
%
\documentclass[
]{article}
\usepackage{amsmath,amssymb}
\usepackage{iftex}
\ifPDFTeX
  \usepackage[T1]{fontenc}
  \usepackage[utf8]{inputenc}
  \usepackage{textcomp} % provide euro and other symbols
\else % if luatex or xetex
  \usepackage{unicode-math} % this also loads fontspec
  \defaultfontfeatures{Scale=MatchLowercase}
  \defaultfontfeatures[\rmfamily]{Ligatures=TeX,Scale=1}
\fi
\usepackage{lmodern}
\ifPDFTeX\else
  % xetex/luatex font selection
\fi
% Use upquote if available, for straight quotes in verbatim environments
\IfFileExists{upquote.sty}{\usepackage{upquote}}{}
\IfFileExists{microtype.sty}{% use microtype if available
  \usepackage[]{microtype}
  \UseMicrotypeSet[protrusion]{basicmath} % disable protrusion for tt fonts
}{}
\makeatletter
\@ifundefined{KOMAClassName}{% if non-KOMA class
  \IfFileExists{parskip.sty}{%
    \usepackage{parskip}
  }{% else
    \setlength{\parindent}{0pt}
    \setlength{\parskip}{6pt plus 2pt minus 1pt}}
}{% if KOMA class
  \KOMAoptions{parskip=half}}
\makeatother
\usepackage{xcolor}
\usepackage[margin=1in]{geometry}
\usepackage{color}
\usepackage{fancyvrb}
\newcommand{\VerbBar}{|}
\newcommand{\VERB}{\Verb[commandchars=\\\{\}]}
\DefineVerbatimEnvironment{Highlighting}{Verbatim}{commandchars=\\\{\}}
% Add ',fontsize=\small' for more characters per line
\usepackage{framed}
\definecolor{shadecolor}{RGB}{248,248,248}
\newenvironment{Shaded}{\begin{snugshade}}{\end{snugshade}}
\newcommand{\AlertTok}[1]{\textcolor[rgb]{0.94,0.16,0.16}{#1}}
\newcommand{\AnnotationTok}[1]{\textcolor[rgb]{0.56,0.35,0.01}{\textbf{\textit{#1}}}}
\newcommand{\AttributeTok}[1]{\textcolor[rgb]{0.13,0.29,0.53}{#1}}
\newcommand{\BaseNTok}[1]{\textcolor[rgb]{0.00,0.00,0.81}{#1}}
\newcommand{\BuiltInTok}[1]{#1}
\newcommand{\CharTok}[1]{\textcolor[rgb]{0.31,0.60,0.02}{#1}}
\newcommand{\CommentTok}[1]{\textcolor[rgb]{0.56,0.35,0.01}{\textit{#1}}}
\newcommand{\CommentVarTok}[1]{\textcolor[rgb]{0.56,0.35,0.01}{\textbf{\textit{#1}}}}
\newcommand{\ConstantTok}[1]{\textcolor[rgb]{0.56,0.35,0.01}{#1}}
\newcommand{\ControlFlowTok}[1]{\textcolor[rgb]{0.13,0.29,0.53}{\textbf{#1}}}
\newcommand{\DataTypeTok}[1]{\textcolor[rgb]{0.13,0.29,0.53}{#1}}
\newcommand{\DecValTok}[1]{\textcolor[rgb]{0.00,0.00,0.81}{#1}}
\newcommand{\DocumentationTok}[1]{\textcolor[rgb]{0.56,0.35,0.01}{\textbf{\textit{#1}}}}
\newcommand{\ErrorTok}[1]{\textcolor[rgb]{0.64,0.00,0.00}{\textbf{#1}}}
\newcommand{\ExtensionTok}[1]{#1}
\newcommand{\FloatTok}[1]{\textcolor[rgb]{0.00,0.00,0.81}{#1}}
\newcommand{\FunctionTok}[1]{\textcolor[rgb]{0.13,0.29,0.53}{\textbf{#1}}}
\newcommand{\ImportTok}[1]{#1}
\newcommand{\InformationTok}[1]{\textcolor[rgb]{0.56,0.35,0.01}{\textbf{\textit{#1}}}}
\newcommand{\KeywordTok}[1]{\textcolor[rgb]{0.13,0.29,0.53}{\textbf{#1}}}
\newcommand{\NormalTok}[1]{#1}
\newcommand{\OperatorTok}[1]{\textcolor[rgb]{0.81,0.36,0.00}{\textbf{#1}}}
\newcommand{\OtherTok}[1]{\textcolor[rgb]{0.56,0.35,0.01}{#1}}
\newcommand{\PreprocessorTok}[1]{\textcolor[rgb]{0.56,0.35,0.01}{\textit{#1}}}
\newcommand{\RegionMarkerTok}[1]{#1}
\newcommand{\SpecialCharTok}[1]{\textcolor[rgb]{0.81,0.36,0.00}{\textbf{#1}}}
\newcommand{\SpecialStringTok}[1]{\textcolor[rgb]{0.31,0.60,0.02}{#1}}
\newcommand{\StringTok}[1]{\textcolor[rgb]{0.31,0.60,0.02}{#1}}
\newcommand{\VariableTok}[1]{\textcolor[rgb]{0.00,0.00,0.00}{#1}}
\newcommand{\VerbatimStringTok}[1]{\textcolor[rgb]{0.31,0.60,0.02}{#1}}
\newcommand{\WarningTok}[1]{\textcolor[rgb]{0.56,0.35,0.01}{\textbf{\textit{#1}}}}
\usepackage{graphicx}
\makeatletter
\def\maxwidth{\ifdim\Gin@nat@width>\linewidth\linewidth\else\Gin@nat@width\fi}
\def\maxheight{\ifdim\Gin@nat@height>\textheight\textheight\else\Gin@nat@height\fi}
\makeatother
% Scale images if necessary, so that they will not overflow the page
% margins by default, and it is still possible to overwrite the defaults
% using explicit options in \includegraphics[width, height, ...]{}
\setkeys{Gin}{width=\maxwidth,height=\maxheight,keepaspectratio}
% Set default figure placement to htbp
\makeatletter
\def\fps@figure{htbp}
\makeatother
\setlength{\emergencystretch}{3em} % prevent overfull lines
\providecommand{\tightlist}{%
  \setlength{\itemsep}{0pt}\setlength{\parskip}{0pt}}
\setcounter{secnumdepth}{-\maxdimen} % remove section numbering
\ifLuaTeX
  \usepackage{selnolig}  % disable illegal ligatures
\fi
\usepackage{bookmark}
\IfFileExists{xurl.sty}{\usepackage{xurl}}{} % add URL line breaks if available
\urlstyle{same}
\hypersetup{
  pdftitle={Coding Challenge 6},
  pdfauthor={Prativa Chhetri and Karamjit Kaur Baryah},
  hidelinks,
  pdfcreator={LaTeX via pandoc}}

\title{Coding Challenge 6}
\author{Prativa Chhetri and Karamjit Kaur Baryah}
\date{2025-03-27}

\begin{document}
\maketitle

{
\setcounter{tocdepth}{2}
\tableofcontents
}
\section{Question 1}\label{question-1}

· reducing the chance of errors from manual repetition.

· Custom functions make your code more organized and reusable.

· If a built-in function changes in future versions of R or a package
update alters behavior, your custom function ensures stability

· Writing your own functions allows you to tailor calculations, data
transformations, or iterations to specific requirements.

\section{Question 2}\label{question-2}

\begin{enumerate}
\def\labelenumi{(\Alph{enumi})}
\tightlist
\item
  Writing a Function in R
\end{enumerate}

A function in R is a reusable block of code that takes inputs
(arguments), processes them, and returns an output. Functions help keep
code organized, reusable, and easier to debug.

Syntax

\$\$my\_function = function(arg1, arg2) \{

result = arg1 + arg2 \# Perform operation

return(result) \# Return output

\}\$\$

How it Works:

· Define the function with function().

· Inside \{\}, write the code to process inputs.

· Use return() to output a value (if omitted, R returns the last
evaluated expression).

· Call the function like this:

\[my_function(3, 5) # should return to 8\]

\begin{enumerate}
\def\labelenumi{(\Alph{enumi})}
\setcounter{enumi}{1}
\tightlist
\item
  Writing a for Loop in R
\end{enumerate}

A for loop in R repeats a block of code for each value in a sequence
(like a vector).

Syntax \$\$ for (i in 1:5) \{

print(i) \# Prints numbers 1 to 5

\} \$\$

How it Works:

· for (i in sequence): Iterates over each value in sequence.

· Inside \{\}, write the code that executes on each iteration.

· The loop stops when all values in sequence are processed.

\$\$squares = c()

for (i in 1:5) \{

squares{[}i{]} = i\^{}2 \# Store squared values

\}

print(squares) \# Returns {[}1{]} 1 4 9 16 25 \$\$

\section{Question 3}\label{question-3}

\begin{Shaded}
\begin{Highlighting}[]
\NormalTok{cities }\OtherTok{\textless{}{-}} \FunctionTok{read.csv}\NormalTok{(}\StringTok{"Cities.csv"}\NormalTok{)}
\FunctionTok{head}\NormalTok{(cities)}
\end{Highlighting}
\end{Shaded}

\begin{verbatim}
##          city  city_ascii state_id state_name county_fips county_name     lat
## 1    New York    New York       NY   New York       36081      Queens 40.6943
## 2 Los Angeles Los Angeles       CA California        6037 Los Angeles 34.1141
## 3     Chicago     Chicago       IL   Illinois       17031        Cook 41.8375
## 4       Miami       Miami       FL    Florida       12086  Miami-Dade 25.7840
## 5     Houston     Houston       TX      Texas       48201      Harris 29.7860
## 6      Dallas      Dallas       TX      Texas       48113      Dallas 32.7935
##        long population density
## 1  -73.9249   18832416 10943.7
## 2 -118.4068   11885717  3165.8
## 3  -87.6866    8489066  4590.3
## 4  -80.2101    6113982  4791.1
## 5  -95.3885    6046392  1386.5
## 6  -96.7667    5843632  1477.2
\end{verbatim}

\section{Question 4}\label{question-4}

\begin{Shaded}
\begin{Highlighting}[]
\NormalTok{haversine\_distance }\OtherTok{\textless{}{-}} \ControlFlowTok{function}\NormalTok{(lat1, lon1, lat2, lon2) \{}
\NormalTok{  rad.lat1 }\OtherTok{\textless{}{-}}\NormalTok{ lat1 }\SpecialCharTok{*}\NormalTok{ pi }\SpecialCharTok{/} \DecValTok{180}
\NormalTok{  rad.lon1 }\OtherTok{\textless{}{-}}\NormalTok{ lon1 }\SpecialCharTok{*}\NormalTok{ pi }\SpecialCharTok{/} \DecValTok{180}
\NormalTok{  rad.lat2 }\OtherTok{\textless{}{-}}\NormalTok{ lat2 }\SpecialCharTok{*}\NormalTok{ pi }\SpecialCharTok{/} \DecValTok{180}
\NormalTok{  rad.lon2 }\OtherTok{\textless{}{-}}\NormalTok{ lon2 }\SpecialCharTok{*}\NormalTok{ pi }\SpecialCharTok{/} \DecValTok{180}
  
\NormalTok{  delta\_lat }\OtherTok{\textless{}{-}}\NormalTok{ rad.lat2 }\SpecialCharTok{{-}}\NormalTok{ rad.lat1}
\NormalTok{  delta\_lon }\OtherTok{\textless{}{-}}\NormalTok{ rad.lon2 }\SpecialCharTok{{-}}\NormalTok{ rad.lon1}
\NormalTok{  a }\OtherTok{\textless{}{-}} \FunctionTok{sin}\NormalTok{(delta\_lat }\SpecialCharTok{/} \DecValTok{2}\NormalTok{)}\SpecialCharTok{\^{}}\DecValTok{2} \SpecialCharTok{+} \FunctionTok{cos}\NormalTok{(rad.lat1) }\SpecialCharTok{*} \FunctionTok{cos}\NormalTok{(rad.lat2) }\SpecialCharTok{*} \FunctionTok{sin}\NormalTok{(delta\_lon }\SpecialCharTok{/} \DecValTok{2}\NormalTok{)}\SpecialCharTok{\^{}}\DecValTok{2}
\NormalTok{  c }\OtherTok{\textless{}{-}} \DecValTok{2} \SpecialCharTok{*} \FunctionTok{asin}\NormalTok{(}\FunctionTok{sqrt}\NormalTok{(a))}
  
\NormalTok{  earth\_radius }\OtherTok{\textless{}{-}} \DecValTok{6378137}  \CommentTok{\# in meters}
\NormalTok{  distance\_km }\OtherTok{\textless{}{-}}\NormalTok{ (earth\_radius }\SpecialCharTok{*}\NormalTok{ c) }\SpecialCharTok{/} \DecValTok{1000}
  
  \FunctionTok{return}\NormalTok{(distance\_km)}
\NormalTok{\}}
\end{Highlighting}
\end{Shaded}

\section{Question 5}\label{question-5}

\begin{Shaded}
\begin{Highlighting}[]
\CommentTok{\# Filter data for Auburn, AL and New York City}
\NormalTok{auburn }\OtherTok{\textless{}{-}}\NormalTok{ cities }\SpecialCharTok{\%\textgreater{}\%} \FunctionTok{filter}\NormalTok{(city }\SpecialCharTok{==} \StringTok{"Auburn"}\NormalTok{)}
\NormalTok{nyc }\OtherTok{\textless{}{-}}\NormalTok{ cities }\SpecialCharTok{\%\textgreater{}\%} \FunctionTok{filter}\NormalTok{(city }\SpecialCharTok{==} \StringTok{"New York"}\NormalTok{)}

\CommentTok{\# Compute distance between Auburn, AL and New York City}
\NormalTok{distance\_nyc\_auburn }\OtherTok{\textless{}{-}} \FunctionTok{haversine\_distance}\NormalTok{(auburn}\SpecialCharTok{$}\NormalTok{lat, auburn}\SpecialCharTok{$}\NormalTok{long, nyc}\SpecialCharTok{$}\NormalTok{lat, nyc}\SpecialCharTok{$}\NormalTok{long)}
\FunctionTok{print}\NormalTok{(distance\_nyc\_auburn)}
\end{Highlighting}
\end{Shaded}

\begin{verbatim}
## [1] 1367.854
\end{verbatim}

\begin{Shaded}
\begin{Highlighting}[]
\NormalTok{results }\OtherTok{\textless{}{-}} \FunctionTok{data.frame}\NormalTok{(}\AttributeTok{City1 =} \FunctionTok{character}\NormalTok{(), }\AttributeTok{City2 =} \FunctionTok{character}\NormalTok{(), }\AttributeTok{Distance\_km =} \FunctionTok{numeric}\NormalTok{(), }\AttributeTok{stringsAsFactors =} \ConstantTok{FALSE}\NormalTok{)}

\ControlFlowTok{for}\NormalTok{ (i }\ControlFlowTok{in} \DecValTok{1}\SpecialCharTok{:}\FunctionTok{nrow}\NormalTok{(cities)) \{}
\NormalTok{  city }\OtherTok{\textless{}{-}}\NormalTok{ cities[i, ]}
  \ControlFlowTok{if}\NormalTok{ (city}\SpecialCharTok{$}\NormalTok{city }\SpecialCharTok{!=} \StringTok{"Auburn"}\NormalTok{) \{}
\NormalTok{    dist }\OtherTok{\textless{}{-}} \FunctionTok{haversine\_distance}\NormalTok{(auburn}\SpecialCharTok{$}\NormalTok{lat, auburn}\SpecialCharTok{$}\NormalTok{long, city}\SpecialCharTok{$}\NormalTok{lat, city}\SpecialCharTok{$}\NormalTok{long)}
\NormalTok{    results }\OtherTok{\textless{}{-}} \FunctionTok{rbind}\NormalTok{(results, }\FunctionTok{data.frame}\NormalTok{(}\AttributeTok{City1 =}\NormalTok{ city}\SpecialCharTok{$}\NormalTok{city, }\AttributeTok{City2 =} \StringTok{"Auburn"}\NormalTok{, }\AttributeTok{Distance\_km =}\NormalTok{ dist))}
\NormalTok{  \}}
\NormalTok{\}}
\FunctionTok{print}\NormalTok{(}\FunctionTok{head}\NormalTok{(results))}
\end{Highlighting}
\end{Shaded}

\begin{verbatim}
##         City1  City2 Distance_km
## 1    New York Auburn   1367.8540
## 2 Los Angeles Auburn   3051.8382
## 3     Chicago Auburn   1045.5213
## 4       Miami Auburn    916.4138
## 5     Houston Auburn    993.0298
## 6      Dallas Auburn   1056.0217
\end{verbatim}

\end{document}
